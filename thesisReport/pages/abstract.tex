\chapter{\abstractname}
Microservice architecture provides various advantages compared to Monolith architecture and thus has gained a lot of attention. However, there are still many aspects of microservices architecture which are not clearly documented and communicated. This research attempts to provide a clear picture regarding various concepts of microservices architecture such as granularity, modeling process and finally create a comprehensive guidelines for implementing microservices.\\
Various keywords from definitions provided by different authors were taken and cateogorized into various conceptual areas. Theses concepts along with the keywords were researched throroughly to understand the concepts of microservices architecture. Additionally, the three important drivers: quality attributes, constraints and principles, are also focussed for creating guidelines.\\
A lot of interesting findings are made. Although, the concept of microservices emphasizes on creating small services, the notion of appropriate granularity is more important and depends upon four basic concepts which are : single responsibility, autonomy, infrastructure capability and business value. Additionally, the quality attributes such as coupling, cohesion etc should also be considered for identification of microservices. A list of basic metrices are created, which can make it easy to qualify microservices.\\
In order to identify microservices, either domain driven design or use case refactoring can be used. Both these approach can be effective but the concept of bounded context in domain driven design identifies autonomous services with single responsibility. Apart from literature, a detail study of the architectural approach used in industry called SAP Hybris is done. A number of interviews conducted with their key personnels give important insight into the process of modeling as well as operating microservices. Again, challenges for implementing microservices as well as the approach to tackle them are done based on the literature and interviews done at SAP Hybris.\\
Finally, all the findings were used to create a detail guidelines for implementing microservices. This is one of the major outcome of the research which dictates how to approach modeling microservices architecture as well as its operational complexities.