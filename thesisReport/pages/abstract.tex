\chapter{\abstractname}
The microservices architecture provides various advantages such as agility, independent scalability etc., as compared to the monolith architecture and thus has gained a lot of attention.However, implementing microservices architecture is still a challenge as many concepts within the microservices architecture including granularity and modeling process are not yet clearly defined and documented. This research attempts to provide a clear understanding of these concepts and finally create a comprehensive guidelines for implementing microservices.\\
Various keywords from definitions provided by different authors are taken and cateogorized into various conceptual areas. These concepts along with the keywords are researched thoroughly to understand the microservices architecture. Additionally, the three important drivers: quality attributes, constraints and principles, are also focussed for creating the guidelines.\\
The findings of this research indicate that eventhough microservices emphasize on the concept of creating small services, the notion of appropriate granularity is more important and depends upon four basic concepts which are : single responsibility, autonomy, infrastructure capability and business value. Additionally, the quality attributes such as coupling, cohesion etc should also be considered for identification of microservices.\\
Furthermore, in order to identify microservices, either the domain driven design approach or the use case refactoring approach can be used. Both these approaches can be effective in identifying microservices, but the concept of bounded context in domain driven design approach identifies autonomous services with single responsibility. Apart from literature, a detail study of the architectural approach used in industry named SAP Hybris is conducted. The interviews conducted with their key personnels has given important insight into the process of modeling as well as operating microservices. Moreover, the challenges for implementing microservices as well as the approach to tackle them are done based on the literature review and the interviews done at SAP Hybris.\\
Finally, the findings are used to create a detail guidelines for implementing microservices. The guidelines captures how to approach modeling microservices architecture as well as its operational complexities.