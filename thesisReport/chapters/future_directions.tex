\chapter{Future Directions}\label{chapter:future_directions}
The current thesis research is conducted on the basis of various academic as well as industrial researches. It would be interesting if the current research could create opportunities for new researches.\\
The Table \ref {tab:quality_of_service/quality_attributes/basic_quality_metrics} listed some basic metrices which can be used to evaluate the quality of microservices. The very first thing which can be done is to find the threshold range that can classify good and bad microservices. A certain number of microservices can be taken as sample from which microservices with high scalability, reusability or other external quality attributes can be filtered out from non performing microservices with low scalability and reusability. Then basic metrices tables can be filled along with their corresponding values for each microservice. These tables can used to find threshold values. The threshold values can be verified further by using them while creating new microservice and checking if their external quality attributes meet the expectations.\\
Another direction of the research can be finding the priority sequence of quality attributes for microservices. From literature, the quality attributes to focus are coupling, cohesion and autonomy. During the interview conducted with SAP Hybris, the quality attributes such as scalability and reusability were given priority when mapping functionalities to microservices. It can be valuable to research further in literature as well as in other industries regarding the priority of quality attributes they choose to identify microservices.\\
Furthermore, the basic metrices table can be leveraged to create a graphical tool which representes various microservices as nodes and connections between them as lines between nodes. It would be benefical if various basic metrices can be evaluated only with \acrshort{API} definitions automatically. Then, the graphical tool can be used to show the various quality attributes using the basic metrices. Depending upon the expected quality, the graphical tool can be used to change the definitiona or \acrshort{API} definition can be changed directly. This can be a great interactive tool for optimizing quality of microservices at design time.\\
Nevertheless, the report can always be used as a base to conduct research on other similar industries as SAP Hybris. It can be interesting to view the result obtained from the study at various different industries.
Given the situation that there are very few literature research done in this area and most of the industries which are using microservices have not openly communicated the whole lifecycle process of microservices, this document can be a very good resource for any company to implement microservices architecture. Additionally, the document can also be a good source for further researches in this area.
