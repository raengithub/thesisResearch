\chapter{Related Work}\label{chapter:related_work}
Microservices is quite new architecture and it is not surprising that there are very few research attempts regarding the overal process of modeling them. Although there are a lot of articles which share the experience of using microservices, only few of them actually share how they achieved the resulting architecture. According to process described in \cite{Levcovitz:2014aa}, firstly database tables are divided into various business areas and then business functionalities and corresponding tables they act upon, are grouped together as microservices. It may only be used in special cases because an assumption is made that there exist a monolith system which has to be broken down into components as microservices and analysing the codebase is one of the necessary steps followed to relate business function with database tables. Furthermore, it does not provide any clear explanation regarding how to break the data into tables handling autonomy. Also, there is no idea about how different quality attributes will be managed when breaking the monolith into various business areas. Another research paper \cite{Bruggemann:2013aa} also share its process of finding microservices but does not provide enough explanation except that the microservices are mapped from product or features of the system.
\\
The approach applied in the current research, as already defined in section \ref{section:context/approach}, takes quality attributes and modeling process into account. It would also be interesting to see the related works on these two topics.\\
Looking back to component oriented architecture can also be helpful to identify some concepts. The research paper \cite{Lee:2001aa} describes process of using coupling and cohesion to indentify individual components. In the process, a single usecase is mapped to a component such that interaction among the objects inside components is decreased. Althoug the whole process may not be used for services but the idea of usecase can be applied.\\
\cite{Ma:2009aa} defines an iterative process evaluating portfolio of services around various quality metrices in order get the better quality services. It takes various quality attributes such as cohesion, coupling, size etc into consideration to find out their overal metrics value. The process is iterated until the services are obtained with satisfied values. The consideration of quality attributes and evaluating them is well thought in this process but the process of coming up with the initial set of services from business domain is not well documented.
Additionally there are a lot of methodology standards based around \acrshort{SOA} to model the services. Some of these methodologies are \acrshort{SOAF}, \acrshort{SOMA}, \acrshort{SOAD} and others. The paper \cite{Ramollari:2016aa} presents a detail list and comparision of these methodologies. Most of them are not yet implemented in industries and others not compatible with agile practices.