\chapter{Conclusion}\label{chapter:conclusion}
% What you learned
% What went well, bad, surprises, insights
% How you'd do it differently?
Software architecture provides a set of guidelines to decompose a system into subsystems, components and modules. Additionally, it defines interaction amongst its individual components. Defining architecture is an important part in software development. So, irrespective of any architecture, there should be precise guidelines to implement the architecture.\\
One of the basic architectural approach is monolithic architecture in which an application is deployed as a single artifact. With this architecture, development, deployment and scaling is simple as long as the application is small. However, as the application gets big and complex, the architecture faces various disadvantages including limited agility, decrease in productivity, longterm commitment to technology stack, limited scalability etc. This is where, a different architectural style 'Microservices' comes into play. In this approach, an application is composed using different independent, autonomous components, each component fulfils a single functionality and can be deployed as well as updated independently. However, there are various concepts related to the definitions of microservices such as collaboration, granularity, mapping of business capability etc that are not clearly communicated. Moreover, there are no proper guidelines how to follow microservices architecture. This is the motivation of the current research. The objective of the research is to clarify various concepts related to microservices and finally create a precise guidelines for implementing microservices.\\
In order to achieve that, various definitions provided by different authors are taken as starting point. A list of keywords and areas are selected from them, which are further investigated during research. Furthermore, quality attributes, constraints and principles are considered as the building block of any architecture. So, these areas are also considered for research to create guidelines.\\
Granularity of microservice is considered as an important concept and is discussed a lot. Granularity is not a one dimensional entity but has three distinct dimensions which are functionality, data and business value. Increase in any of these dimension will increase granularity. Also, it is important to consider the concept of 'Appropriate' granularity rather than 'Minimum' size. Appropriate granularity of any microservice is achieved by three basic concepts: \textbf{Single Responsibility Principle}, \textbf{Autonomy}, and \textbf{Infrastructure capability}.
\\
Adding to that, other quality attributes such as coupling, cohesion, autonomy etc are as important as granularity. For that purpose, a list of basic metrics are created to evaluate each quality attributes easily. The quality attributes are not mutually exclusive but affect eachother. The relationship among them is clarified so that it becomes easy to determine the appropriate tradeoffs when necessary.\\
Another important concept which is not clearly communicated is the identification of microservices and mapping of business capability to microservices. Two different approaches are discussed which are \textbf{Using UseCase Refactoring} and  \textbf{Using Domain Driven Design}.\\
UseCase Refactoring utilizes use cases to breakdown a problem domain and refactor them based on various concepts such as similarity in functionality, similarity on entities they act etc. A discrete set of rules for usecase refactoring are also present. On the other hand, domain driven design uses the concept of ubiquitous language and bounded context to break down a problem domain. A detailed step for each phase is also presented. Furthermore, a case study is broken down into microservices using each approach.\\
Usecase Refactoring is comparatively easier since architects and developers are more familiar with the concept. Domain Driven Design on the other hand is a complex and iterative procedure. Furthermore, Usecase Refactoring considers an entity as a single source of truth for all sub-domains in entire system. Using this approach does not necessarily produce autonomous microservices but focus more on functionality. However, domain driven design using the concept of ubiquitous language and bounded context creates autonomous microservices with single responsibility.\\
After conducting research along various literature, another important part is determining the process followed in industry for which SAP Hybris is chosen. Firstly, various available documents are studied which suggested vision and principles as being the driving force for the process. In order to understand indepth, interviews are conducted with various key personnels related to \acrshort{YaaS}. A major outcome of the interview is that, in addition to factors such as autonomy, infrastructure capability and Single Responsibility Principle, \textbf{Business Value} of the microservices plays significant role when identifying microservices  and defining the optimum granularity. Finally, the workflow followed during deployment of microservices in SAP Hybris is also studied in order to clarify the operational approach together with modeling process.\\
Understanding constraints is another important driver of architecture. The approach to handle various constraints can be helpful in defining guidelines. Microservices architecture has various advantages such as strong modularity, agility, independent deployment capability however also presents various challenges for maintaining these advantages. The challenges can be kept into three distinct groups: 
\textbf{Distributed System Complexity}
\textbf{Integration}, and
\textbf{Operational Complexity}.\\
Different challenges along each group are discussed. Additionally, the various techniques which can be used to tackle each  are listed based on the litereature. Finally, the techniques used in SAP Hybris to handle each challenge are also mentioned.\\
With all the studies performed, the process of microservices architecture can be disected along two phases:
\textbf{Modeling Phase} and 
\textbf{Operation Phase}.\\
During modeling phase, problem domain is studied, various internal quality attributes along with business value and infrastructure capability are considered. Whereas during operational phase, various challenges are tackled. The effectiveness of modeling phase is visible across various external quality attributes in this phase. Again, principles are major driving factor for creating architectural guidelines. The principles along both modeling and operation phase are listed based on the previous findings of current research. Finally, for each principles, a set of guidelines are presented. The guidelines are one of the major outcomes of the research.
